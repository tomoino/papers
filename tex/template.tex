\documentclass{jsarticle}
\bibliographystyle{jplain}
\usepackage[dvipdfmx]{graphicx}
\usepackage{amsmath}
\usepackage{amsfonts}
\usepackage{bm}
\usepackage{wrapfig}
\usepackage{MnSymbol}
\usepackage{scalefnt}
\usepackage{here}
\title{\vspace{-3cm}論文読みまとめ\\論文タイトル}
\author{井上智裕}
\date{2020年11月17日}

\makeatletter
\renewcommand{\thefigure}{\thesection.\arabic{figure}}
\@addtoreset{figure}{section}
\renewcommand{\thetable}{\thesection.\arabic{table}}
\@addtoreset{table}{section}
\renewcommand{\theequation}{\thesection.\arabic{equation}}
\@addtoreset{equation}{section}

\newcommand{\argmax}{\mathop{\rm arg~max}\limits}

\begin{document}
\maketitle
\vspace{-1cm}
\section{導入}
\subsection{扱う問題}
画像処理か?音声認識か?
\subsection{問題意識}
どこに問題意識を感じているのか?既存手法では何が足りないのか?

\section{理論}
どのようなアイデア・ロジック・仮定で問題を解決しようとしているか?なぜそれで問題が解決できるのか?
\subsection{定式化・アルゴリズム}
アイデアをどのように定式に落とし込んでいるか?それぞれの式は何を意味しているのか?

\section{実験}
どんな実験設定か?どんな点で他の手法より良くなったか?

\section{結論・展望}
他に改善するとしたらどこか?

%% \begin{table}[htbp]
%%   \begin{center}
%%     \caption{何かの表}
%%     \begin{tabular}{|c|c|c|c|c|} \hline
%%       ユーザー & 映画1 & 映画2 & 映画3 & 映画4 \\ \hline 
%%       A & 5 & 3 &   &  \\
%%       B &  & 2 &  & 5 \\
%%       C &  &  & 4 & 3 \\ 
%%       D & 4 &  & 3 & \\
%%       E & 4 & 3 &  & 2\\ \hline
%%     \end{tabular}
%%     \label{tab:tab1}
%%   \end{center}
%% \end{table}

%% \begin{figure}[htbp]
%%   \begin{center}
%%     \includegraphics[clip,width=12.0cm]{/path/to/hoge.png}
%%     \caption{何かの図}
%%     \label{fig:fig1}
%%   \end{center}
%% \end{figure}

% \begin{eqnarray}
%   score(h_t,\bar h_s) = \begin{cases}
%     h_t^T\bar h_s & dot \\
%     h_t^TW_a\bar h_s & general \\
%     v_a^T \tanh(W_a [h_t;\bar h_s]) & concat \\
%   \end{cases}
% \end{eqnarray}

\end{document}
